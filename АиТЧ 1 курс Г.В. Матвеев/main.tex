\documentclass[a4paper, 12pt]{report}
\usepackage[T2A]{fontenc} 
\usepackage[utf8]{inputenc}
\usepackage[english,russian]{babel} 

\usepackage{amsmath,amsfonts,amssymb,amsthm,mathtools}

\usepackage[left=2cm,right=2cm,top=2cm,bottom=2cm,bindingoffset=0cm]{geometry}

\newtheorem*{theorem}{Теорема}
\newenvironment{Proof}
{\par\noindent{$\blacklozenge$}}
{\hfill$\scriptstyle\boxtimes$}

\begin{document}
\section{Прямая сумма подпространств}
Пусть $W_1$, $W_2$ - подпространства.\\
$W_1 \oplus W_2$ - сумма называется прямой, если $W_1 \cap W_2 = \vec0$\\
Справедливо и следующее: $W_1 \oplus W_2 \oplus ... \oplus W_k$ называется прямой, если $W_i \cap \sum\limits_{i \neq j} W_j = \vec 0$
\begin{theorem}
    $$dim(W_1 \oplus W_2) = dimW_1 + dimW_2$$
\end{theorem}
\begin{Proof}
    По теореме о сумме подпространств $$dim(W_1+W_2)=dimW_1+dimW_2-dim(W_1 \cap W_2)$$
    А так как $W_1 \cap W_2 = \vec0$, то $dim(W_1 \cap W_2)=0$.
\end{Proof}\\
\textbf{Следствие:}
    $$dim(W_1 \oplus W_2 \oplus ... \oplus W_k) = dimW_1 + dimW_2 + ... + dimW_k$$
\begin{theorem}
    Если $W \subset V_n \Rightarrow V_n = W \oplus U$, где $U$ - подпространство.
\end{theorem} 
\begin{Proof}
    \begin{enumerate}
        \item $W = \vec0 \Rightarrow U = V_n, V_n = \vec0 \oplus V_n$
        \item $W = V_n \Rightarrow U = \vec0, V_n = V_n + \vec0$\\
        Оба равенства справедливы, так как $\vec0 \cap V_n = \vec0$
        \item Рассмотрим нетривиальный случай:
        $$W = L(v_1, v_2,...,v_r), \quad 0<r<n$$
        $$U = L(v_{r+1},v_{r+2},...,v_n)$$
        Возьмем произвольный вектор $x$, не нарушая общности:
        $$x=(\alpha_1v_1+...+\alpha_rv_r)+(\alpha_{r+1}v_{r+1}+...+\alpha_nv_n) \Rightarrow x=W+U$$
        Докажем, что $W \cap U = \vec0$.\\
        Пусть $x \in W \cap U$.
        $$x=\alpha_1v_1+...+\alpha_rv_r=\alpha_{r+1}v_{r+1}+...+\alpha_nv_n \Rightarrow \forall \alpha_i = 0 \Rightarrow x = 0 \Rightarrow W \cap U = \vec0$$
    \end{enumerate}
\end{Proof}\\
\textbf{Следствие}\\
Каждое пространство раскладывается в прямую сумму $n$ одномерных подпространств.
$$V_n=L(e_1) \oplus L(e_2) \oplus ... \oplus L(e_n)$$ 
$e_1,e_2, ... ,e_n$-базис.\\
То есть любой вектор раскладываетя по базису:
$$x=\alpha_1e_1+\alpha_2e_2+ ... + \alpha_ne_n$$

\section{Критерий совместности системы линейных уравнений}
Система линейных алгебраических уравнений совместна тогда и только тогда, когда ранг матрицы коэффицентов равен рангу расширенной матрицы.
\begin{Proof}
   $$ \begin{cases}
        a_{11}x_1+a_{12}x_2+...+a_{1n}x_n=b_1\\
        a_{21}x_1+a_{22}x_2+...+a_{2n}x_n=b_2\\
        \dotfill\\
        a_{m1}x_1+a_{m2}x_2+...+a_{mn}x_n=b_m
    \end{cases}$$
    \begin{center}
    $\begin{pmatrix}
    a_{11} & a_{12} & \dots & a_{1n}\\
    a_{21} & a_{22} & \dots & a_{2n}\\
    \vdots & \vdots & \ddots & \vdots\\
    a_{m1} & a_{m2} & \dots & a_{mn}
    \end{pmatrix}$
    Матрица коэффицентов А.\\
    \end{center}
    \begin{center}
     $\begin{pmatrix}
    a_{11} & a_{12} & \dots & a_{1n} & \vline & b_1\\
    a_{21} & a_{22} & \dots & a_{2n} & \vline & b_2\\
    \vdots & \vdots & \ddots & \vdots & \vline & \vdots\\
    a_{m1} & a_{m2} & \dots & a_{mn} & \vline & b_m
    \end{pmatrix}$
    $\widetilde{A} = (A/B)$ расширенная матрица.   
    \end{center}
    Система совместна $\Leftrightarrow rankA=rank\widetilde{A}$. \\
    $\Rightarrow$ Пусть система совместна с решением $(j_1,j_2, \dots, j_n)$\\
    $$\begin{pmatrix}
    a_{11}\\
    a_{21}\\
    \vdots\\
    a_{n1}
    \end{pmatrix} \cdot j_1+
    \begin{pmatrix}
    a_{12}\\
    a_{22}\\
    \vdots\\
    a_{n2}
    \end{pmatrix} \cdot j_2
    + \dots +
    \begin{pmatrix}
    a_{1n}\\
    a_{2n}\\
    \vdots\\
    a_{nn}
    \end{pmatrix} \cdot j_n = 
    \begin{pmatrix}
    b_1\\
    b_2\\
    \vdots\\
    b_n
    \end{pmatrix} \eqno (1)$$
    Это значит, что при добавлении столбца свободных членов базис не изменился, так как новый столбец выражается через старый. Следовательно $rankA=rank\widetilde{A}$.\\
    $\Leftarrow$ Базисный минор  матрицы $A$ есть базисный минор матрицы $\widetilde{A}$, так как $rankA=rank\widetilde{A}$. Следовательно столбец свободных членов 
    $ \begin{pmatrix}
    b_1\\
    b_2\\
    \vdots\\
    b_n
    \end{pmatrix}$ выражается через базисные столбцы по принципу (1). Коэффиценты остальных столбцов равны 0. И тогда полученные коэффиценты будут являться решением системы.
\end{Proof}

\subsection*{Решение системы линейных алгебраических уравнений с помощью критерия}
\begin{enumerate}
    \item Нахождение базисного минора матрицы $A$ методом окаямления минора.
    \item Проверяем условие $rankA=rank\widetilde{A}$ методом окаямления миноров.
    \item Отбрасываем все небазисные строки.
    \item Базисные неизвестные оставляем слева, а свободные переносим вправо.
\end{enumerate}
\begin{equation}
    \begin{cases}
    $$a_{11}x_1+a_{12}x_2+ \dots + a_{1r}x_r = b_1-a_{1, r+1}x_{r+1}- \dots -a_{1n}x_n$$\\
    $$\dotfill$$\\
    $$a_{n1}x_1+a_{n2}x_2+ \dots + a_{nr}x_r = b_n-a_{n, r+1}x_{r+1}- \dots -a_{nn}x_n$$
    \end{cases}
\end{equation}
Полученную систему рассматриваем как крамеровскую.\\
$M=
\begin{vmatrix}
a_{11} & \dots & a_{1r}\\
\dots & \ddots & \dots\\
a_{r1} & \dots & a_{rr}
\end{vmatrix} \neq 0$
\begin{equation*}
    \begin{cases}
    $$x_1=f_1(x_{r+1},\dots,x_n)$$\\
    \dotfill\\
    $$x_r=f_r(x_{r+1},\dots,x_n)$$
    \end{cases}
\end{equation*}
\section{Однородные системы линейных уравнений}
\begin{equation}
    \begin{cases}
    $$a_{11}x_1+a_{12}x_2+\dots+a_{1n}x_n=0$$\\
    $$a_{21}x_1+a_{22}x_2+\dots+a_{2n}x_n=0$$\\
    \dotfill\\
    $$a_{m1}x_1+a_{m2}x_2+\dots+a_{mn}x_n=0$$
    \end{cases}
\end{equation}
 $A=
\begin{pmatrix}
a_{11} & a_{12} & \dots & a_{1n}\\
a_{21} & a_{22} & \dots & a_{2n}\\
\vdots & \vdots & \ddots & \vdots\\
a_{m1} & a_{m2} & \dots & a_{mn}
\end{pmatrix}$ - матрица системы.
$X=
\begin{pmatrix}
x_1\\
x_2\\
\vdots\\
x_n
\end{pmatrix}
$ - столбец неизвестных.\\
\begin{center}
$0=
\begin{pmatrix}
0\\
0\\
\vdots\\
0
\end{pmatrix}$ - столбец нулей.
\end{center}
$$\textbf{AX=0}$$
\begin{theorem}
    Решения однородной системы линейных уравнений образуют векторное пространство, размерность которого $dimW=n-r$ ($n$-число неизвестных, $r$- ранг системы, $r=rankA=rank(A|0)$
\end{theorem}
\begin{Proof}
   Докажем, что это пространство. Вспомним необходимые критерии:
   $$W_1, W_2  \in W \Rightarrow W_1+W_2 \in W$$
   $$W_1 \in W \Rightarrow \lambda W_1 \in W$$
   $X_1$ - конкретный набор, $X_1=
\begin{pmatrix}
x_1\prime\\
x_2\prime\\
\vdots\\
x_n\prime
\end{pmatrix}$\\
$$AX_1=0, AX_2=0 \Rightarrow A(X_1+X_2)=0$$
$$AX_1=0 \Rightarrow \lambda AX_1=0$$
Перенесем свободные неизвестные в системе в левую сторону.
\begin{equation}
    \begin{cases}
    $$a_{11}x_1+a_{12}x_2+ \dots + a_{1r}x_r = b_1-a_{1, r+1}x_{r+1}- \dots -a_{1n}x_n$$\\
    $$\dotfill$$\\
    $$a_{n1}x_1+a_{n2}x_2+ \dots + a_{nr}x_r = b_n-a_{n, r+1}x_{r+1}- \dots -a_{nn}x_n$$
    \end{cases}
\end{equation}
Базисный минор $M=
\begin{vmatrix}
a_{11} & \dots & a_{1r}\\
\dots & \ddots & \dots\\
a_{r1} & \dots & a_{rr}
\end{vmatrix} \neq 0$\\
$x_1,\dots,x_n$-базисные.\\
$x_{r+1},\dots,x_n$-свободные.\\
Выражаем базисные неизвестные через свободные по правилу Крамера или Гаусса:
\begin{equation*}
    \begin{cases}
    $$x_1=f_1(x_{r+1},\dots,x_n)$$\\
    \dotfill\\
    $$x_r=f_r(x_{r+1},\dots,x_n)$$
    \end{cases}
\end{equation*}
Найдем базисные решения:\\
Передадим значения 
$$c_1=(c_{11},c_{12},\dots,c_{1r},1,0,\dots,0)$$
$$c_2=(c_{21},c_{22},\dots,c_{2r},0,1,\dots,0)$$
$$c_{n-r}=(c_{n-r,1},c_{n-r,2},\dots,c_{n-r,r},0,0,\dots,1)$$
Переменные, которым были переданы значения 0 и 1 являются базисными. Векторы являются линейно независимыми благодаря этим переменным.\\
Докажем, что любое решение выражается через базис.\\
$$(\gamma_1,\dots,\gamma_r,\gamma_{r+1},\dots,\gamma_n)-\gamma_{r+1}c_1-\dots-\gamma_nc_{n-r}=(\gamma_1c_1,\gamma_2c_2,\dots,\gamma_nc_{n-r})$$
Значит все решения выражаются через базис.\\
\textit{Базисные решения ОСЛУ называются фундаментальной системой решений.}
\end{Proof}
\subsection*{Решение неоднородной системы через однородную}
$AX=B$ - неоднородная, $AX=0$ - однородная.
$$\left.
  \begin{array}{ccc}
    AX=B \\
    AY=0 \\
  \end{array}
\right\}=A(X+Y)=AX+AY=B+0=B$$
\begin{enumerate}
    \item Разность 2-ух решений неоднородной будет решением однородной.
    \item Если от решения неоднородной отнять фиксированное решение неоднородной, то получится решение однородной.
    $$AX-AX_0=B-B=0$$
    \item Произвольное решение неоднородной можно получить, добавляя к фиксированному решению некоторые решения однородной.
\end{enumerate}

\section{Линейные преобразования векторных пространств}
Отображение $\varphi$ $V \rightarrow V$ (само в себя) называется линейным, если
\begin{enumerate}
    \item Образ суммы равен сумме образов:
    $$\varphi(a+b)=\varphi(a)+\varphi(b)$$
    \item При умножении вектора на скаляр его образ умножается на этот же скаляр:
    $$\varphi(\lambda a)=\lambda \varphi(a)$$
\end{enumerate}
Если $\varphi: V \rightarrow W$, то $\varphi$ - линейное отображение.\\
Свойства:
\begin{enumerate}
    \item Образ линейной комбинации равен такой же линейной комбинации образов (под действием линейного преобразования)
    $$\varphi(\lambda_1a_1+\lambda_2a_2+\dots+\lambda_na_n=\lambda_1\varphi(a_1)+\lambda_2\varphi(a_2)+\dots+\lambda_n\varphi(a_n)$$
    \item Преобразование $\vec0$
    $$\varphi(\vec0)=\vec0$$
    $$\varphi(\vec0)=\varphi(\vec0\cdot\vec a)=0\varphi(\vec a)=\vec0$$
    \item Вынесение минуса
    $$\varphi(-\vec a)=-\varphi(\vec a)$$
    \item Линейное преобразование переводит линейно зависимые в линейно зависимые с такими же скалярами.
\end{enumerate}
\begin{theorem}
    Любое линейное преобразование вполне определяется своими значениями на базисных векторах и эти значения могут быть любыми.
    \begin{Proof}
       Пусть $e_1,e_2,\dots,e_n$ - базис, $a_1,a_2,\dots,a_n$- системы векторов.\\
       Возьмем функцию $\varphi$ такую, что:
       \begin{equation*}
           \begin{cases}
                  $$\varphi(e_1)=a_1$$\\
                  $$\varphi(e_2)=a_2$$\\
                  \dotfill\\
                  $$\varphi(e_n)=a_n$$
           \end{cases}
       \end{equation*}
       Докажем, что такое пространство существует:
       $$x=x_1e_1+x_2e_2+\dots+x_ne_n$$
       $$\varphi(x)=x_1a_1+x_2a_2+\dots+x_na_n$$
       Докажем, что оно линейное:
       $$y=y_1e_1+y_2e_2+\dots+y_ne_n$$
       \begin{itemize}
           \item $\varphi(x+y)=(x_1+y_1)a_1+(x_2+y_2)a_2+\dots+(x_n+y_n)a_n=x_1a_1+y_1a_1+\dots+x_na_n+y_na_n=(x_1a_1+x_2a_2+\dots+x_na_n)+(y_1a_1+y_2a_2+\dots+y_na_n)=\varphi(x)+\varphi(y)$
           \item $\varphi(\lambda x)=\lambda x_1a_1+\lambda x_2a_2+\dots+\lambda x_na_n = \lambda \varphi(x)$
       \end{itemize}
       Докажем, что единственное:\\
       Пусть существует 
       \begin{equation*}
       \begin{cases}
              $$\psi(e_1)=a_1$$\\
              $$\psi(e_2)=a_2$$\\
              \dotfill\\
              $$\psi(e_n)=a_n$$
       \end{cases}    
       \end{equation*} с такими же свойствами.
       $$\psi(x)=\psi(x_1e_1+x_2e_2+\dots+x_ne_n)=x_1\psi(e1)+x_2\psi(e_2)+\dots+x_n\psi(e_n)=x_1a_1+x_2a_2+\dots+x_na_n=\varphi(x)$$
    \end{Proof}
\end{theorem}
\section{Операции над линейными преобразованиями}
$f, \varphi$ - линейные преобразования векторного пространства $V$.
\begin{enumerate}
    \item $(f+\varphi)(\lambda_1 x_1 + \lambda_2 x_2)=f(\lambda_1 x_1 + \lambda_2 x_2) + \varphi(\lambda_1 x_1 + \lambda_2 x_2)=f(\lambda_1 x_1) + f(\lambda_2 x_2) + \varphi(\lambda_1 x_1) + \varphi(\lambda_2 x_2) = \lambda_1f(x_1)+\lambda_2f(x_2)+\lambda_1\varphi(x_1)+\lambda_2\varphi(x_2)=\lambda_1(f(x_1)+\varphi(x_1))+\lambda_2(f(x_2)+\varphi(x_2))=\lambda_1(f+\varphi)(x_1)+\lambda(f+\varphi)(x_2)$
    \item $(\lambda f)(\lambda_1 x_1 + \lambda_2 x_2)=(\lambda f)(\lambda_1 x_1)+(\lambda f)(\lambda_2 x_2)=\lambda f(\lambda_1 x_1)+\lambda f(\lambda_2 x_2)=\lambda(f(\lambda_1 x_1)+\lambda(f(\lambda_2 x_2))=\lambda f(\lambda_1x_1+\lambda_2x_2)$
    \item $(f\varphi)(\lambda_1x_1+\lambda_2x_2)=f(\varphi(\lambda_1x_1+\lambda_2x_2))=f(\varphi(\lambda_1x_1)+\varphi(\lambda_2x_2))=f)\lambda_1\varphi(x_1)+\lambda_2\varphi(x_2))=\lambda_1f(\varphi(x_1))+\lambda_2f(\varphi(x_2))=\lambda_1(f\varphi)(x_1)+\lambda_2(f\varphi)(x_2)$
\end{enumerate}
\section{Ранг и дефект линейного преобразования}
$\varphi:V\rightarrow V$\\
$ker \varphi = \{x|\varphi(x)=\vec 0\}$ - ядро (деффект).\\
$Im \varphi = \varphi(v)=\{\varphi(x)|x \in V\}$ - образ (ранг).\\
\textbf{Пример1}\\
Рассмотрим функцию $\sin(x)$. Функция синуса не является линейной, в чем легко убедиться ($\sin(a+b) \neq \sin a + \sin b$), однако для нее можно определить ядро и образ. Таким образом
$$ker \sin = {\pi n}$$
$$Im \sin = [-1,1]$$
\textbf{Пример2}\\
Тождественное преобразование - $\varphi(v)=v \quad \forall v \in V$
$$ker \varphi = \vec 0$$
$$Im \varphi = V$$
\textbf{Пример3}\\
Возьмем прямую $l$ и плоскость $P$, где $l \perp P$.
$$\varphi(\vec a)=\vec pr a$$
$$Im \varphi = l = V_1$$
$$ker \varphi = P = V_2$$
\begin{theorem}
    Ядро и образ линейного преобразования - подпространства исходного векторного пространства.
\end{theorem}
\begin{Proof}
   \begin{enumerate}
       \item $w_1,w_2 \in ker \varphi \Rightarrow \varphi(w_1)=\varphi(w_2)=\vec 0$
       $\varphi(w_1+w_2)=\varphi(w_1)+\varphi(w_2)=\vec 0+ \vec 0=\vec 0 \Rightarrow w_1+w_2 \in ker \varphi$
       \item $\lambda \varphi(w)=\lambda \vec 0=\vec 0 \Rightarrow \lambda \varphi \in ker \varphi$
       \item $\varphi(w_1),\varphi(w_2) \in Im \varphi$\\
       $\varphi(w_1)+\varphi(w_2)=\varphi(w_1+w_2) \in Im \varphi$
       \item $\lambda \varphi(w_1)=\varphi(\lambda w_1) \in Im \varphi$
   \end{enumerate}
\end{Proof}
$$d = dim \; ker \varphi$$
$$r = rank \varphi = dim \; Im \varphi$$
$\varphi$ - нулевое преобразование:
$$d=n, \quad r=0$$
$\varphi$ - тождественное преобразование:
$$d=0, \quad r=n$$
$\varphi$ - проектирование векторов:
$$d=2, \quad r=1$$
\begin{theorem}
    Сумма ранга и дефекта равняется размерности пространства.
\end{theorem}
\begin{Proof}
   Рассмотрим образ $\varphi(V)$. Пусть базис $\varphi(V):\varphi(\varphi(l_1),\varphi(l_2),\dots,\varphi(l_r))$\\
   Докажем, что $$V_n=L(l_1,l_2,\dots,l_r) \oplus ker\varphi$$
   $$n=r+d$$
   \begin{enumerate}
       \item $l_1,l_2,\dots,l_r$ - линейно независимы.\\
       По свойству линейное преобразование сохраняет зависимость. Если бы $l_1,l_2,\dots,l_r$ были зависимы, то и $\varphi(l_1),\varphi(l_2),\dots,\varphi(l_r)$ были бы зависимы, но это базис, значит не зависимы.
       \item $\vec v \in V_n = \vec x \in L(l_1,l_2,\dots,l_r) + \vec y \in ker \varphi$\\
       $$\varphi(V) = \alpha_1\varphi(l_1)+\alpha_2\varphi(l_2)+\dots+\alpha_r\varphi(l_r)$$
       $$\varphi(v-\alpha_1 l_1-\dots-\alpha_r l_r)=\vec 0 \Rightarrow v-\alpha_1 l_1-\dots-\alpha_r l_r = y \in ker \varphi$$
       $$v = \alpha_1 l_1-\dots-\alpha_r l_r + y = x + y$$
       \item $L \cap ker \varphi = \vec 0$\\
       Пусть $x \in L \cap ker \varphi$.\\
       $x = \alpha_1 l_1+\dots+\alpha_r l_r$\\
       $\varphi(x) = \vec 0, \quad \varphi(x) = \varphi(\alpha_1 l_1+\dots+\alpha_r l_r) = \varphi(\alpha_1 l_1)+\varphi(\alpha_2 l_2)+\dots+\varphi(\alpha_r l_r)=\alpha_1\varphi(l_1)+\alpha_2\varphi(l_2)+\dots+\alpha_r\varphi(l_r) = \vec 0 \Rightarrow \alpha_1=\alpha_2=\dots=\alpha_r=0 \Rightarrow x=\vec 0$
   \end{enumerate}
\end{Proof}
\end{document}
