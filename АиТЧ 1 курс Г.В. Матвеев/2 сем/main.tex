\documentclass[a4paper, 12pt]{article}
\usepackage[T2A]{fontenc} 
\usepackage[utf8]{inputenc}
\usepackage[english,russian]{babel} 


\usepackage{amsmath,amsfonts,amssymb,amsthm,mathtools}

\usepackage[left=2cm,right=2cm,top=2cm,bottom=2cm,bindingoffset=0cm]{geometry}

\newtheorem*{theorem}{Теорема}
\newtheorem*{corollary}{Следствие}
\newenvironment{Proof}
{\par\noindent{$\blacklozenge$}}
{\hfill$\scriptstyle\boxtimes$}

\usepackage[normalem]{ulem}
\usepackage[unicode]{hyperref}

%доп команды
\newcommand{\rank}{\operatorname{rank}}
\renewcommand{\Im}{\operatorname{Im}}
\renewcommand{\Re}{\operatorname{Re}}
\renewcommand{\ker}{\operatorname{ker}}

%Красивые греческие буквы
\usepackage{upgreek}
\renewcommand{\alpha}{\upalpha}
\renewcommand{\beta}{\upbeta}
\renewcommand{\gamma}{\upgamma}
\renewcommand{\delta}{\updelta}
\renewcommand{\varphi}{\upvarphi}
\renewcommand{\epsilon}{\upepsilon}
\renewcommand{\psi}{\uppsi}


\title{\vspace{6.5cm}\textbf{\Huge{Алгебра и Теория чисел}}\\Конспект по 2 семестру 
	специальности «прикладная информатика»\\(лектор Г. В. Матвеев)}
\date{}
\begin{document}
	\maketitle
	\newpage
	\tableofcontents{}
	\newpage
\section{Прямая сумма подпространств}
Пусть $W_1$, $W_2$ --- подпространства.\\\\
$\bullet$ \textit{$W_1 \oplus W_2$ --- сумма называется \textbf{прямой}, если $W_1 \cap W_2 = \vec0$}.\\\\
Справедливо и следующее: $W_1 \oplus W_2 \oplus ... \oplus W_k$ называется прямой, если $W_i \cap \sum\limits_{i \neq j} W_j = \vec 0$
\begin{theorem}
    $$\dim(W_1 \oplus W_2) = \dim W_1 + \dim W_2$$
\end{theorem}
\begin{Proof}
    По теореме о сумме подпространств $$\dim(W_1+W_2)=\dim W_1+\dim W_2-\dim(W_1 \cap W_2)$$
    А так как $W_1 \cap W_2 = \vec0$, то $\dim(W_1 \cap W_2)=0$.
\end{Proof}
\begin{corollary}
	$$\dim(W_1 \oplus W_2 \oplus \ldots \oplus W_k) = \dim W_1 + \dim W_2 + \ldots + \dim W_k$$
\end{corollary}
\begin{theorem}
    Если $W \subset V_n \Rightarrow V_n = W \oplus U$, где $U$ --- подпространство.
\end{theorem} 
\begin{Proof}
    \begin{enumerate}
        \item $W = \vec0 \Rightarrow U = V_n$, $V_n = \vec0 \oplus V_n$
        \item $W = V_n \Rightarrow U = \vec0$, $V_n = V_n + \vec0$\\
        Оба равенства справедливы, так как $\vec0 \cap V_n = \vec0$
        \item Рассмотрим нетривиальный случай:
        $$W = L(v_1, v_2,...,v_r), \quad 0<r<n$$
        $$U = L(v_{r+1},v_{r+2},...,v_n)$$
        Возьмем произвольный вектор $x$, не нарушая общности:
        $$x=(\alpha_1v_1+...+\alpha_rv_r)+(\alpha_{r+1}v_{r+1}+...+\alpha_nv_n) \Rightarrow x=W+U$$
        Докажем, что $W \cap U = \vec0$.\\
        Пусть $x \in W \cap U$.
        $$x=\alpha_1v_1+...+\alpha_rv_r=\alpha_{r+1}v_{r+1}+...+\alpha_nv_n \Rightarrow \forall \alpha_i = 0 \Rightarrow x = 0 \Rightarrow W \cap U = \vec0$$
    \end{enumerate}
\end{Proof}
\begin{corollary}
	Каждое пространство раскладывается в прямую сумму $n$ одномерных подпространств.
	$$V_n=L(e_1) \oplus L(e_2) \oplus ... \oplus L(e_n)$$ 
\end{corollary}
$e_1,e_2, ... ,e_n$-базис.\\
То есть любой вектор раскладываетя по базису:
$$x=\alpha_1e_1+\alpha_2e_2+ ... + \alpha_ne_n$$

\section{Критерий совместности системы линейных уравнений}
\begin{theorem}
	Система линейных алгебраических уравнений совместна тогда и только тогда, когда ранг матрицы коэффицентов равен рангу расширенной матрицы.
\end{theorem}
\begin{Proof}
	Рассмотрим систему алгебраических уравнений
   $$ \begin{cases}
        a_{11}x_1+a_{12}x_2+...+a_{1n}x_n=b_1\\
        a_{21}x_1+a_{22}x_2+...+a_{2n}x_n=b_2\\
        \dotfill\\
        a_{m1}x_1+a_{m2}x_2+...+a_{mn}x_n=b_m
    \end{cases}$$
    \begin{center}
    $A = \begin{pmatrix}
    a_{11} & a_{12} & \dots & a_{1n}\\
    a_{21} & a_{22} & \dots & a_{2n}\\
    \vdots & \vdots & \ddots & \vdots\\
    a_{m1} & a_{m2} & \dots & a_{mn}
    \end{pmatrix}$
    --- матрица коэффицентов $A$.\\
    \end{center}
    \begin{center}
     $\begin{pmatrix}
    a_{11} & a_{12} & \dots & a_{1n} & \vline & b_1\\
    a_{21} & a_{22} & \dots & a_{2n} & \vline & b_2\\
    \vdots & \vdots & \ddots & \vdots & \vline & \vdots\\
    a_{m1} & a_{m2} & \dots & a_{mn} & \vline & b_m
    \end{pmatrix} =
    \widetilde{A} = (A|B)$ --- расширенная матрица.   
    \end{center}
    Система совместна $\Leftrightarrow \rank A=\rank\widetilde{A}$. \\
    $\Rightarrow$ Пусть система совместна с решением $(j_1,j_2, \dots, j_n)$\\
    $$\begin{pmatrix}
    a_{11}\\
    a_{21}\\
    \vdots\\
    a_{n1}
    \end{pmatrix} \cdot j_1+
    \begin{pmatrix}
    a_{12}\\
    a_{22}\\
    \vdots\\
    a_{n2}
    \end{pmatrix} \cdot j_2
    + \dots +
    \begin{pmatrix}
    a_{1n}\\
    a_{2n}\\
    \vdots\\
    a_{nn}
    \end{pmatrix} \cdot j_n = 
    \begin{pmatrix}
    b_1\\
    b_2\\
    \vdots\\
    b_n
    \end{pmatrix} \eqno (1)$$
    Это значит, что при добавлении столбца свободных членов базис не изменился, так как новый столбец выражается через старый. Следовательно, $\rank A=\rank\widetilde{A}$.\\\\
    $\Leftarrow$ Базисный минор  матрицы $A$ есть базисный минор матрицы $\widetilde{A}$, так как $rankA=rank\widetilde{A}$. Следовательно, столбец свободных членов 
    $ \begin{pmatrix}
    b_1\\
    b_2\\
    \vdots\\
    b_n
    \end{pmatrix}$ выражается через базисные столбцы по принципу (1). Коэффиценты остальных столбцов равны 0. И тогда полученные коэффиценты будут являться решением системы.
\end{Proof}

\subsection*{Решение системы линейных алгебраических уравнений с помощью критерия}
\begin{enumerate}
    \item Нахождение базисного минора матрицы $A$ методом окаямления минора.
    \item Проверяем условие $\rank A=\rank\widetilde{A}$ методом окаямления миноров.
    \item Отбрасываем все небазисные строки.
    \item Базисные неизвестные оставляем слева, а свободные переносим вправо.
\end{enumerate}

   $$ \begin{cases}
    a_{11}x_1+a_{12}x_2+ \dots + a_{1r}x_r = b_1-a_{1, r+1}x_{r+1}- \dots -a_{1n}x_n\\
    \dotfill\\
    a_{n1}x_1+a_{n2}x_2+ \dots + a_{nr}x_r = b_n-a_{n, r+1}x_{r+1}- \dots -a_{nn}x_n
    \end{cases}$$

Полученную систему рассматриваем как крамеровскую.\\
$$M=
\begin{vmatrix}
a_{11} & \dots & a_{1r}\\
\dots & \ddots & \dots\\
a_{r1} & \dots & a_{rr}
\end{vmatrix} \neq 0$$\\
\begin{equation*}
    \begin{cases}
    $$x_1=f_1(x_{r+1},\dots,x_n)$$\\
    \dotfill\\
    $$x_r=f_r(x_{r+1},\dots,x_n)$$
    \end{cases}
\end{equation*}
\section{Однородные системы линейных уравнений}
Рассмотрим однородную систему линейных уравнений
    $$\begin{cases}
    a_{11}x_1+a_{12}x_2+\dots+a_{1n}x_n=0,\\
    a_{21}x_1+a_{22}x_2+\dots+a_{2n}x_n=0\\
    \dotfill\\
    a_{m1}x_1+a_{m2}x_2+\dots+a_{mn}x_n=0
    \end{cases}\eqno(1)$$
Где
 $A=
\begin{pmatrix}
a_{11} & a_{12} & \dots & a_{1n}\\
a_{21} & a_{22} & \dots & a_{2n}\\
\vdots & \vdots & \ddots & \vdots\\
a_{m1} & a_{m2} & \dots & a_{mn}
\end{pmatrix}$ --- матрица системы,
$X=
\begin{pmatrix}
x_1\\
x_2\\
\vdots\\
x_n
\end{pmatrix}
$ --- столбец неизвестных.\\
\begin{center}
$0=
\begin{pmatrix}
0\\
0\\
\vdots\\
0
\end{pmatrix}$ --- столбец нулей.
\end{center}
Тогда систему (1) можно записать в матричном виде как
$$\textbf{AX=0}$$
\begin{theorem}
    Решения однородной системы линейных уравнений образуют векторное пространство, размерность которого $\dim W=n-r$ ($n$ --- число неизвестных, $r$ --- ранг системы, $r=\rank A=\rank(A|0)$.
\end{theorem}
\begin{Proof}
   Докажем, что это пространство. Вспомним необходимые критерии:
   $$W_1, W_2  \in W \Rightarrow W_1+W_2 \in W$$
   $$W_1 \in W \Rightarrow \lambda W_1 \in W$$
   Пусть $X_1$ --- конкретный набор, $X_1=
\begin{pmatrix}
x_1\prime\\
x_2\prime\\
\vdots\\
x_n\prime
\end{pmatrix}$. Тогда выполняются свойства
$$AX_1=0,\ AX_2=0 \Rightarrow A(X_1+X_2)=0$$
$$AX_1=0 \Rightarrow \lambda AX_1=0$$
Перенесем свободные неизвестные в системе в левую сторону.
    $$\begin{cases}
    a_{11}x_1+a_{12}x_2+ \ldots + a_{1r}x_r = b_1-a_{1, r+1}x_{r+1}- \ldots -a_{1n}x_n\\
    \dotfill\\
    a_{n1}x_1+a_{n2}x_2+ \ldots + a_{nr}x_r = b_n-a_{n, r+1}x_{r+1}- \ldots -a_{nn}x_n
    \end{cases}\eqno(3)$$
Базисный минор для этой системы $$M=
\begin{vmatrix}
a_{11} & \dots & a_{1r}\\
\dots & \ddots & \dots\\
a_{r1} & \dots & a_{rr}
\end{vmatrix} \neq 0$$
Где неизвестные $x_1,\dots,x_n$ --- базисные, а
$x_{r+1},\dots,x_n$ --- свободные.\\
Выражаем базисные неизвестные через свободные по правилу Крамера или Гаусса:
\begin{equation*}
    \begin{cases}
    $$x_1=f_1(x_{r+1},\dots,x_n)$$\\
    \dotfill\\
    $$x_r=f_r(x_{r+1},\dots,x_n)$$
    \end{cases}
\end{equation*}
Найдем базисные решения. Для этого
передадим значения 
$$c_1=(c_{11},c_{12},\dots,c_{1r},1,0,\dots,0)$$
$$c_2=(c_{21},c_{22},\dots,c_{2r},0,1,\dots,0)$$
$$c_{n-r}=(c_{n-r,1},c_{n-r,2},\dots,c_{n-r,r},0,0,\dots,1)$$
Переменные, которым были переданы значения 0 и 1, являются базисными. Векторы являются линейно независимыми благодаря этим переменным.\\\\
Докажем, что любое решение выражается через базис.
$$(\gamma_1,\dots,\gamma_r,\gamma_{r+1},\dots,\gamma_n)-\gamma_{r+1}c_1-\ldots-\gamma_nc_{n-r}=(\gamma_1c_1,\gamma_2c_2,\dots,\gamma_nc_{n-r})$$
Значит все решения выражаются через базис.
\end{Proof}\\\\
$\bullet$ \textit{Базисные решения ОСЛУ называются \textbf{фундаментальной системой решений}.}
\subsection*{Решение неоднородной системы через однородную}
Будем обозначать $AX=B$ --- \textbf{неоднородная система}, $AX=0$ --- \textbf{однородная система}.
$$\left.
  \begin{array}{ccc}
    AX=B \\
    AY=0 \\
  \end{array}
\right\}=A(X+Y)=AX+AY=B+0=B$$
\begin{enumerate}
    \item Разность 2-ух решений неоднородной системы будет решением однородной.
    \item Если от решения неоднородной системы отнять фиксированное решение неоднородной системы, то получится решение однородной системы.
    $$AX-AX_0=B-B=0$$
    \item Произвольное решение неоднородной системы можно получить, добавляя к фиксированному решению некоторые решения однородной системы.
\end{enumerate}
\section{Линейные преобразования векторных пространств}
$\bullet$ \textit{Отображение $\varphi: V \rightarrow V$ (само в себя) называется \textbf{линейным}, если}
\begin{enumerate}
    \item \textit{Образ суммы равен сумме образов:}
    $$\varphi(a+b)=\varphi(a)+\varphi(b)$$
    \item \textit{При умножении вектора на скаляр его образ умножается на этот же скаляр:}
    $$\varphi(\lambda a)=\lambda \varphi(a)$$
\end{enumerate}
Если $\varphi: V \rightarrow W$, то $\varphi$ --- линейное отображение.\\\\
\textit{\textbf{Свойства линейного преобразования:}}
\begin{enumerate}
    \item \textit{Образ линейной комбинации равен такой же линейной комбинации образов (под действием линейного преобразования)}
    $$\varphi(\lambda_1a_1+\lambda_2a_2+\dots+\lambda_na_n=\lambda_1\varphi(a_1)+\lambda_2\varphi(a_2)+\dots+\lambda_n\varphi(a_n)$$
    \item \textit{Преобразование} $\vec0$
    $$\varphi(\vec0)=\vec0$$
    $$\varphi(\vec0)=\varphi(\vec0\cdot\vec a)=0\cdot \varphi(\vec a)=\vec0$$
    \item \textit{Вынесение минуса}
    $$\varphi(-\vec a)=-\varphi(\vec a)$$
    \item \textit{Линейное преобразование переводит линейно зависимые векторы в линейно зависимые с такими же скалярами.}
\end{enumerate}
\begin{theorem}
    Любое линейное преобразование вполне определяется своими значениями на базисных векторах и эти значения могут быть любыми.
\end{theorem}
    \begin{Proof}
       Пусть $e_1,e_2,\dots,e_n$ --- базис, $a_1,a_2,\dots,a_n$ --- системы векторов.\\
       Возьмем функцию $\varphi$ такую, что:
       \begin{equation*}
           \begin{cases}
                  $$\varphi(e_1)=a_1$$\\
                  $$\varphi(e_2)=a_2$$\\
                  \dotfill\\
                  $$\varphi(e_n)=a_n$$
           \end{cases}
       \end{equation*}
       Докажем, что такое пространство существует:
       $$x=x_1e_1+x_2e_2+\dots+x_ne_n$$
       $$\varphi(x)=x_1a_1+x_2a_2+\dots+x_na_n$$
       Докажем, что оно линейное:
       $$y=y_1e_1+y_2e_2+\dots+y_ne_n$$
       \begin{itemize}
           \item $\varphi(x+y)=(x_1+y_1)a_1+(x_2+y_2)a_2+\ldots+(x_n+y_n)a_n=x_1a_1+y_1a_1+\ldots+x_na_n+y_na_n=(x_1a_1+x_2a_2+\ldots+x_na_n)+(y_1a_1+y_2a_2+\ldots+y_na_n)=\varphi(x)+\varphi(y);$
           \item $\varphi(\lambda x)=\lambda x_1a_1+\lambda x_2a_2+\ldots+\lambda x_na_n = \lambda \varphi(x).$
       \end{itemize}
       Докажем, что единственное:\\
       Пусть существует 
       \begin{equation*}
       \begin{cases}
              $$\psi(e_1)=a_1$$\\
              $$\psi(e_2)=a_2$$\\
              \dotfill\\
              $$\psi(e_n)=a_n$$
       \end{cases}    
       \end{equation*} с такими же свойствами. Тогда
       $$\psi(x)=\psi(x_1e_1+x_2e_2+\dots+x_ne_n)=x_1\psi(e1)+x_2\psi(e_2)+\dots+x_n\psi(e_n)=x_1a_1+x_2a_2+\dots+x_na_n=\varphi(x)$$
    \end{Proof}

\section{Операции над линейными преобразованиями}
Пусть $f, \varphi$ --- линейные преобразования векторного пространства $V$.
\begin{enumerate}
    \item \textbf{Сумма линейных преобразований:} $$f(x) + \varphi(x) = (f+\varphi)(x),\ \forall x \in V.$$
    \begin{Proof}
    	$(f+\varphi)(\lambda_1 x_1 + \lambda_2 x_2)=f(\lambda_1 x_1 + \lambda_2 x_2) + \varphi(\lambda_1 x_1 + \lambda_2 x_2)=f(\lambda_1 x_1) + f(\lambda_2 x_2) + \varphi(\lambda_1 x_1) + \varphi(\lambda_2 x_2) = \lambda_1f(x_1)+\lambda_2f(x_2)+\lambda_1\varphi(x_1)+\lambda_2\varphi(x_2)=\lambda_1(f(x_1)+\varphi(x_1))+\lambda_2(f(x_2)+\varphi(x_2))=\lambda_1(f+\varphi)(x_1)+\lambda_2(f+\varphi)(x_2)$.
    \end{Proof} 
    \item \textbf{Произведение на скаляр линейного преобразования:} $$(\lambda f)(x) =  \lambda f(x),\ \forall x \in V.$$
    \begin{Proof}
    	$(\lambda f)(\lambda_1 x_1 + \lambda_2 x_2)=(\lambda f)(\lambda_1 x_1)+(\lambda f)(\lambda_2 x_2)=\lambda f(\lambda_1 x_1)+\lambda f(\lambda_2 x_2)=\lambda(f(\lambda_1 x_1)+\lambda(f(\lambda_2 x_2))=\lambda f(\lambda_1x_1+\lambda_2x_2)$.
    \end{Proof} 
    \item \textbf{Композиция линейных преобразований:} $$(f\circ \varphi)(x) = f(\varphi(x)),\ \forall x \in V.$$
    \begin{Proof}
    	$(f\varphi)(\lambda_1x_1+\lambda_2x_2)=f(\varphi(\lambda_1x_1+\lambda_2x_2))=f(\varphi(\lambda_1x_1)+\varphi(\lambda_2x_2))=f)\lambda_1\varphi(x_1)+\lambda_2\varphi(x_2))=\lambda_1f(\varphi(x_1))+\lambda_2f(\varphi(x_2))=\lambda_1(f\varphi)(x_1)+\lambda_2(f\varphi)(x_2)$.
    \end{Proof} 
\end{enumerate}
\section{Ранг и дефект линейного преобразования}
Пусть $\varphi:V\rightarrow V$ --- линейное преобразование.\\\\
$\bullet$ \textit{Множество $\ker \varphi = \{x\ |\ \varphi(x)=\vec 0\}$ --- \textbf{ядро} линейного преобразования.}\\
$\dim \ker \varphi$ - \textbf{дефект} линейного преобразования (размерность ядра).\\\\
$\bullet$ \textit{Множество $\Im \varphi = \varphi(v)=\{\varphi(x)\ |\ x \in V\}$ --- \textbf{образ} линейного преобразования.}\\
$\dim \Im \varphi$ - \textbf{ранг} линейного преобразования (размерность образа).\\\\
\textbf{Пример 1}\\
Рассмотрим функцию $\sin(x)$. Функция синуса не является линейной, в чем легко убедиться ($\sin(a+b) \neq \sin a + \sin b$), однако для нее можно определить ядро и образ. Таким образом
$$\ker (\sin) = {\pi n}$$
$$\Im (\sin) = [-1,1]$$
\textbf{Пример 2}\\
Тождественное преобразование - $\varphi(v)=v \quad \forall v \in V$
$$\ker( \varphi) = \vec 0$$
$$\Im (\varphi) = V$$
\textbf{Пример 3}\\
Возьмем прямую $l$ и плоскость $P$, где $l \perp P$.
$$\varphi(\vec a)=\vec pr a$$
$$\Im (\varphi) = l = V_1$$
$$\ker (\varphi) = P = V_2$$
\begin{theorem}
    Ядро и образ линейного преобразования --- подпространства исходного векторного пространства.
\end{theorem}
\begin{Proof}
	Проверим выполнимость свойств:
   \begin{enumerate}
       \item $w_1,w_2 \in \ker (\varphi) \Rightarrow \varphi(w_1)=\varphi(w_2)=\vec 0$
       $\varphi(w_1+w_2)=\varphi(w_1)+\varphi(w_2)=\vec 0+ \vec 0=\vec 0 \Rightarrow w_1+w_2 \in \ker (\varphi)$
       \item $\lambda \varphi(w)=\lambda \vec 0=\vec 0 \Rightarrow \lambda \varphi \in \ker (\varphi)$
       \item $\varphi(w_1),\varphi(w_2) \in \Im (\varphi)$\\
       $\varphi(w_1)+\varphi(w_2)=\varphi(w_1+w_2) \in \Im (\varphi)$
       \item $\lambda \varphi(w_1)=\varphi(\lambda w_1) \in \Im (\varphi)$
   \end{enumerate}
\end{Proof}\\
$\bullet$ \textit{Размерность ядра --- \textbf{дефект}. Будем обозначать} $d = \dim( \ker (\varphi))$.\\\\
$\bullet$ \textit{Размерность образа --- \textbf{ранг}. Будем обозначать} $r = \rank \varphi = \dim  (\Im (\varphi))$.\\\\
Тогда $\varphi$ --- \textbf{нулевое преобразование}, если $d=n$, $r=0$.\\
$\varphi$ --- \textbf{тождественное преобразование}, если $d=0$, $r=n$.\\
$\varphi$ --- \textbf{проектирование векторов}, если $d=2$, $r=1$.
\begin{theorem}
    Сумма ранга и дефекта равняется размерности пространства.
\end{theorem}
\begin{Proof}
   Рассмотрим образ $\varphi(V)$. Пусть базис $\varphi(V):\varphi(\varphi(l_1),\varphi(l_2),\dots,\varphi(l_r))$\\
   Докажем, что $$V_n=L(l_1,l_2,\dots,l_r) \oplus \ker(\varphi)$$
   $$n=r+d$$
   \begin{enumerate}
       \item $l_1,l_2,\dots,l_r$ --- линейно независимы.\\
       По свойству линейное преобразование сохраняет зависимость. Если бы $l_1,l_2,\dots,l_r$ были зависимы, то и $\varphi(l_1),\varphi(l_2),\dots,\varphi(l_r)$ были бы зависимы, но это базис, значит не зависимы.
       \item $\vec v \in V_n = \vec x \in L(l_1,l_2,\dots,l_r) + \vec y \in ker \varphi$\\
       $$\varphi(V) = \alpha_1\varphi(l_1)+\alpha_2\varphi(l_2)+\\dots+\alpha_r\varphi(l_r)$$
       $$\varphi(v-\alpha_1 l_1-\ldots-\alpha_r l_r)=\vec 0 \Rightarrow v-\alpha_1 l_1-\ldots-\alpha_r l_r = y \in \ker \varphi$$
       $$v = \alpha_1 l_1-\ldots-\alpha_r l_r + y = x + y$$
       \item $L \cap ker \varphi = \vec 0$\\
       Пусть $x \in L \cap \ker \varphi$.\\
       $x = \alpha_1 l_1+\ldots+\alpha_r l_r$\\
       $\varphi(x) = \vec 0, \quad \varphi(x) = \varphi(\alpha_1 l_1+\ldots+\alpha_r l_r) = \varphi(\alpha_1 l_1)+\varphi(\alpha_2 l_2)+\ldots+\varphi(\alpha_r l_r)=\alpha_1\varphi(l_1)+\alpha_2\varphi(l_2)+\ldots+\alpha_r\varphi(l_r) = \vec 0 \Rightarrow \alpha_1=\alpha_2=\ldots=\alpha_r=0 \Rightarrow x=\vec 0$
   \end{enumerate}
\end{Proof}

\section{Матрица линейного преобразования}
Пусть $V$ --- векторное пространство с базисом $e_1, e_2, \dots, e_n$.
$$x \in V, \quad x = x_1e_1+x_2e_2+\dots+x_ne_n$$
$\bullet$ Пусть $\varphi: V \rightarrow V$ - \textbf{линейное преобразование} векторного пространства $V$.\\
Подействуем этим преобразованием поочередно на все базисные векторы и полученные векторы выразим через базис:
$$\begin{cases}
     \varphi(e_1)=\alpha_{11}e_1+\alpha_{21}e_2+\dots+\alpha_{n1}e_n\\  
     \varphi(e_2)=\alpha_{12}e_1+\alpha_{22}e_2+\dots+\alpha_{n2}e_n\\ 
     \dotfill\\
     \varphi(e_n)=\alpha_{1n}e_1+\alpha_{2n}e_2+\dots+\alpha_{nn}e_n\\ 
\end{cases}\eqno(1)$$
\begin{center}
Матрица $A = 
    \begin{pmatrix}
    \alpha_{11} & \alpha_{12} & \dots & \alpha_{1n}\\
    \alpha_{21} & \alpha_{22} & \dots & \alpha_{2n}\\
    \vdots & \vdots & \ddots & \vdots\\
    \alpha_{n1} & \alpha_{n2} & \dots & \alpha_{nn}
    \end{pmatrix}$ --- матрица линейного преобразования $\varphi$.
\end{center}
$\bullet$ Столбцами матрицы линейного преобразования являются координаты образов базисных векторов.\\\\
Рассмотрим пример:\\
На вектор $x = x_1e_1+x_2e_2+\dots+x_ne_n$ подействуем линейным преобразованием.
$$\varphi(x) = x_1\varphi(e_1)+x_2\varphi(e_2)+\dots+x_n\varphi(e_n)$$
$\varphi(e_1), \varphi(e_2), \dots, \varphi(e_n)$ --- столбцы матрицы $A$.\\
Систему (1) можно переписать следующим образом:\\
Пусть $e = (e_1, e_2, \dots, e_n)$, тогда
$$(e_1, e_2, \dots, e_n)A=(\varphi(e_1), \varphi(e_2), \dots, \varphi(e_n))$$
$$\varphi(e)=eA$$
Вектор $x$ запишем как $x=eX$, где 
$X=
\begin{pmatrix}
x_1\\
x_2\\
\vdots\\
x_n
\end{pmatrix}$.
Тогда линейное преобразование вектора $x$ примет вид:
$$\varphi(x)=\varphi(e)X$$
$$\varphi(x)=eAX$$
Это говорит о том, что $X \xrightarrow{\varphi} AX \sim \varphi(X)=AX$.\\
\textbf{Теорема.}
    \begin{enumerate}
        \item \textit{При сложении линейных преобразований их матрицы в данном базисе складываются.}
        \item \textit{При умножении линейных преобразований их матрицы в данном базисе умножаются.}
        \item \textit{При умножении линейного преобразования на скаляр его матрица умножается на тот же скаляр.}
    \end{enumerate}

\begin{Proof}
Пусть $V$ --- векторное пространство с базисом $e_1, e_2, \dots, e_n$.\\
И пусть $f, \varphi$ --- линейные преобразования.\\
Подействовав этими линейными преобразованиями на базис $V$ получим следующие системы:
$$\begin{cases}
     f(e_1)=\alpha_{11}e_1+\alpha_{21}e_2+\dots+\alpha_{n1}e_n\\  
     f(e_2)=\alpha_{12}e_1+\alpha_{22}e_2+\dots+\alpha_{n2}e_n\\ 
     \dotfill\\
     f(e_n)=\alpha_{1n}e_1+\alpha_{2n}e_2+\dots+\alpha_{nn}e_n\\ 
\end{cases}\eqno(1)$$

$$\begin{cases}
     \varphi(e_1)=\beta_{11}e_1+\beta_{21}e_2+\dots+\beta_{n1}e_n\\  
     \varphi(e_2)=\beta_{12}e_1+\beta_{22}e_2+\dots+\beta_{n2}e_n\\ 
     \dotfill\\
     \varphi(e_n)=\beta_{1n}e_1+\beta_{2n}e_2+\dots+\beta_{nn}e_n\\ 
\end{cases}\eqno(2)$$\\
Запишем матрицы линейных преобразований для $f, \varphi$:\\
\begin{center}
$A = 
\begin{pmatrix}
\alpha_{11} & \alpha_{12} & \dots & \alpha_{1n}\\
\alpha_{21} & \alpha_{22} & \dots & \alpha_{2n}\\
\vdots & \vdots & \ddots & \vdots\\
\alpha_{n1} & \alpha_{n2} & \dots & \alpha_{nn}
\end{pmatrix}$ --- матрица линейного преобразования $f$.
\end{center}
\begin{center}
$B = 
\begin{pmatrix}
\beta_{11} & \beta_{12} & \dots & \beta_{1n}\\
\beta_{21} & \beta_{22} & \dots & \beta_{2n}\\
\vdots & \vdots & \ddots & \vdots\\
\beta_{n1} & \beta_{n2} & \dots & \beta_{nn}
\end{pmatrix}$ --- матрица линейного преобразования $\varphi$.
\end{center}
   \begin{enumerate}
       \item Сложим почленно строки систем (1) и (2).
        $$\begin{cases}
        f(e_1)+\varphi(e_1)=(\alpha_{11}+\beta_{11})e_1+(\alpha_{21}+\beta_{21})e_2+\dots+(\alpha_{n1}+\beta_{n1})e_n\\
        f(e_2)+\varphi(e_2)=(\alpha_{12}+\beta_{12})e_1+(\alpha_{22}+\beta_{22})e_2+\dots+(\alpha_{n2}+\beta_{n2})e_n\\
        \dotfill\\
        f(e_n)+\varphi(e_n)=(\alpha_{1n}+\beta_{1n})e_1+(\alpha_{2n}+\beta_{2n})e_2+\dots+(\alpha_{nn}+\beta_{nn})e_n\\ 
        \end{cases}$$
        Отсюда получим матрицу линейного преобразования $f+\varphi$:\\
        $$\begin{pmatrix}
        \alpha_{11}+\beta_{11} & \alpha_{12}+\beta_{12} & \dots & \alpha_{1n}+\beta_{1n}\\
        \alpha_{21}+\beta_{21} & \alpha_{22}+\beta_{22} & \dots & \alpha_{2n}+\beta_{2n}\\
        \vdots & \vdots & \ddots & \vdots\\
        \alpha_{n1}+\beta_{n1} & \alpha_{n2}+\beta_{n2} & \dots & \alpha_{nn}+\beta_{nn}\\
        \end{pmatrix} = A + B$$
        \item Будем рассматривать умножение линейных преобразований как компзицию отображений $\varphi(f(e))$.
        Подействуем линейным преобразованием $f$ на базисные векторы:
        $$\begin{cases}
        f(e_1)=\alpha_{11}e_1+\alpha_{21}e_2+\dots+\alpha_{n1}e_n\\  
        f(e_2)=\alpha_{12}e_1+\alpha_{22}e_2+\dots+\alpha_{n2}e_n\\ 
        \dotfill\\
        f(e_n)=\alpha_{1n}e_1+\alpha_{2n}e_2+\dots+\alpha_{nn}e_n\\ 
        \end{cases}\eqno(1)$$
        На полученные векторы подействуем линейным преобразованием $\varphi$:
        $$\begin{cases}
        \varphi(f(e_1))=\beta_{11}f(e_1)+\beta_{21}f(e_2)+\dots+\beta_{n1}f(e_n)\\  
        \varphi(f(e_2))=\beta_{12}f(e_2)+\beta_{22}f(e_2)+\dots+\beta_{n2}f(e_n)\\
        \varphi(f(e_n))=\beta_{1n}f(e_1)+\beta_{2n}f(e_2)+\dots+\beta_{nn}f(e_n)\\
        \end{cases}$$
        Подставим в полученную систему уравнения системы (1):\\\\
        $\begin{cases}
        \varphi(f(e_1))=\beta_{11}(\alpha_{11}e_1+\alpha_{21}e_2+\dots+\alpha_{n1}e_n)+\beta_{21}(\alpha_{12}e_1+\alpha_{22}e_2+\dots+\alpha_{n2}e_n)+\dots\\  
       \varphi(f(e_2))=\beta_{12}(\alpha_{11}e_1+\alpha_{21}e_2+\dots+\alpha_{n1}e_n)+\beta_{22}(\alpha_{12}e_1+\alpha_{22}e_2+\dots+\alpha_{n2}e_n)+\dots\\ 
        \varphi(f(e_n))=\beta_{1n}(\alpha_{11}e_1+\alpha_{21}e_2+\dots+\alpha_{n1}e_n)+\beta_{2n}(\alpha_{12}e_1+\alpha_{22}e_2+\dots+\alpha_{n2}e_n)+\dots\\ 
        \end{cases}$\\\\
        Раскроем скобки:\\\\
        $\begin{cases}
        \varphi(f(e_1))=\beta_{11}\alpha_{11}e_1+\beta_{11}\alpha_{21}e_2+\dots+\beta_{11}\alpha_{n1}e_n+\beta_{21}\alpha_{12}e_1+\beta_{21}\alpha_{22}e_2+\dots+\beta_{21}\alpha_{n2}e_n+\dots\\  
       \varphi(f(e_2))=\beta_{12}\alpha_{11}e_1+\beta_{12}\alpha_{21}e_2+\dots+\beta_{12}\alpha_{n1}e_n+\beta_{22}\alpha_{12}e_1+\beta_{22}\alpha_{22}e_2+\dots+\beta_{22}\alpha_{n2}e_n+\dots\\ 
        \varphi(f(e_n))=\beta_{1n}\alpha_{11}e_1+\beta_{1n}\alpha_{21}e_2+\dots+\beta_{1n}\alpha_{n1}e_n+\beta_{2n}\alpha_{12}e_1+\beta_{2n}\alpha_{22}e_2+\dots+\beta_{2n}\alpha_{n2}e_n+\dots\\ 
        \end{cases}$\\\\
        Сгрупируем подобные слогаемые:\\\\
        $\begin{cases}
        \varphi(f(e_1))=(\beta_{11}\alpha_{11}+\beta_{21}\alpha_{12}+\dots)e_1+(\beta_{11}\alpha_{21}+\beta_{21}\alpha_{22}+\dots)e_2+\dots\\  
       \varphi(f(e_2))=(\beta_{12}\alpha_{11}+\beta_{22}\alpha_{12}+\dots)e_1+(\beta_{12}\alpha_{21}+\beta_{22}\alpha_{22}+\dots)e_2+\dots\\ 
        \varphi(f(e_n))=(\beta_{1n}\alpha_{11}+\beta_{2n}\alpha_{12}+\dots)e_1+(\beta_{1n}\alpha_{21}+\beta_{2n}\alpha_{22}+\dots)e_2+\dots\\  
        \end{cases}$\\
        Запишем координаты векторов в матрицу линейного преобразования:\\
        $\begin{pmatrix}
        \beta_{11}\alpha_{11}+\beta_{21}\alpha_{12}+\dots & \beta_{12}\alpha_{11}+\beta_{22}\alpha_{12}+\dots & \dots & \beta_{1n}\alpha_{11}+\beta_{2n}\alpha_{12}+\dots\\
       \beta_{11}\alpha_{21}+\beta_{21}\alpha_{22}+\dots & \beta_{12}\alpha_{21}+\beta_{22}\alpha_{22}+\dots&\dots&
       \beta_{2n}\alpha_{21}+\beta_{2n}\alpha_{22}+\dots\\
        \vdots & \vdots & \ddots & \vdots\\
        \beta_{11}\alpha_{n1}+\beta_{21}\alpha_{n2}+\dots & \beta_{12}\alpha_{n1}+\beta_{22}\alpha_{n2}+\dots & \dots&
        \beta_{1n}\alpha_{n1}+\beta_{2n}\alpha_{n2}+\dots\\
        \end{pmatrix}=A\cdot B$
        \item Умножим каждую строку системы (1) на произвольный скаляр $\gamma$:
        $$\begin{cases}
        \gamma f(e_1)=\gamma\alpha_{11}e_1+\gamma \alpha_{21}e_2+\dots+\gamma \alpha_{n1}e_n\\  
        \gamma f(e_2)=\gamma \alpha_{12}e_1+\gamma \alpha_{22}e_2+\dots+\gamma \alpha_{n2}e_n\\ 
        \dotfill\\
        \gamma f(e_n)=\gamma \alpha_{1n}e_1+\gamma \alpha_{2n}e_2+\dots+\gamma \alpha_{nn}e_n\\ 
        \end{cases}$$
        Получаем матрицу линейного преобразования $\gamma f$:
        \begin{center}
        $\begin{pmatrix}
        \gamma \alpha_{11} & \gamma \alpha_{12} & \dots & \gamma \alpha_{1n}\\
        \gamma \alpha_{21} & \gamma \alpha_{22} & \dots & \gamma \alpha_{2n}\\
        \vdots & \vdots & \ddots & \vdots\\
        \gamma \alpha_{n1} & \gamma \alpha_{n2} & \dots & \gamma \alpha_{nn}
        \end{pmatrix} = \gamma A$
        \end{center}
   \end{enumerate}
\end{Proof}
\begin{theorem}
    Ранг линейного преобразования равен рангу его матрицы.
\end{theorem}
\begin{Proof}
$$\rank \varphi = \dim \varphi(V)$$
Так как образ есть \textit{линейная оболочка} $L(\varphi(e_1), \varphi(e_2), \dots, \varphi(e_n))$, то
$$\dim \varphi(V) = \dim L(\varphi(e_1), \varphi(e_2), \dots, \varphi(e_n)) = rank(\varphi(e_1), \varphi(e_2), \dots, \varphi(e_n)) = \rank A$$
$$\rank \varphi = \rank A$$
\end{Proof}\\
\textbf{Пример 1}\\
$\varphi(x) = \vec 0, \quad \forall x$ --- нулевое преобразование.
$$\begin{cases}
     \varphi(e_1)=0e_1+0e_2+\dots+0e_n\\  
     \varphi(e_2)=0e_1+0e_2+\dots+0e_n\\ 
     \dotfill\\
     \varphi(e_n)=0e_1+0e_2+\dots+0e_n\\ 
\end{cases}$$
\begin{center}
    $A = \begin{pmatrix}
    0 & 0 & \dots & 0\\
    0 & 0 & \dots & 0\\
    \vdots & \vdots & \ddots & \vdots\\
    0 & 0 & \dots & 0
    \end{pmatrix}$
    --- матрица нулевого преобразования.\\
    \end{center}
\textbf{Пример 2}\\
$\varphi(x) = x, \quad \forall x$ --- тождественное преобразование.
$$\begin{cases}
     \varphi(e_1)=1e_1+0e_2+\dots+0e_n\\  
     \varphi(e_2)=0e_1+1e_2+\dots+0e_n\\ 
     \dotfill\\
     \varphi(e_n)=0e_1+0e_2+\dots+1e_n\\ 
\end{cases}$$
\begin{center}
    $A = \begin{pmatrix}
    1 & 0 & \dots & 0\\
    0 & 1 & \dots & 0\\
    \vdots & \vdots & \ddots & \vdots\\
    0 & 0 & \dots & 1
    \end{pmatrix}$
    --- матрица тождественного преобразования.\\
    \end{center}    
\textbf{Пример 3}
\begin{center}
    $A = \begin{pmatrix}
    \cos \alpha & -\sin \alpha\\
    \sin \alpha & \cos \alpha\\
    \end{pmatrix}$
    --- матрица угла поворота системы координат на угол $\alpha$.\\
    \end{center}
    $\bullet$ Биективное (взаимооднозначное) линейное преобразование называется \textbf{автоморфизмом}.\\
    Если $\varphi: V \rightarrow V$ --- линейное преобразование, то $\varphi$ --- автоморфизм $\Leftrightarrow$ $\varphi$ --- биекция.
    \begin{theorem}
        Линейное преобразование $\varphi$ --- автоморфизм $\Leftrightarrow$ его матрица невырожденная.
    \end{theorem}
    \begin{Proof}
    $$\varphi(X)=AX$$
    $\Rightarrow$ Предположим, что $|A| = 0$ (т.е. матрица вырожденная).\\
    Тогда $AX=0$ $\Rightarrow$ система уравнений линейного преобразования имеет несколько решений и ноль имеет несколько прообразов, чего быть не может.\\\\
    $\Leftarrow$ Имеем, что $|A| \neq 0$ (т.е. матрица невырожденная).\\
    Значит для $AX=B$ имеется только одно решение по правилу Крамера $\Rightarrow$ $\varphi$ --- биекция.
    \end{Proof}
\end{document}
