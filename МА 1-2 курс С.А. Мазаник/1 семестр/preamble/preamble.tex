\documentclass[a4paper, 12pt]{report} % стиль документа
\usepackage[T2A]{fontenc} 
\usepackage[utf8]{inputenc}
\usepackage[english,russian]{babel}
\usepackage{upgreek}
\usepackage{graphicx}
\usepackage[normalem]{ulem}
% поддержка русского языка в документе

\usepackage{amsmath,amsfonts,amssymb,amsthm,mathtools} % библиотеки AMS

\usepackage[left=2cm,right=2cm,top=2cm,bottom=2cm,bindingoffset=0cm]{geometry} % выравнивание по краям

\usepackage[unicode]{hyperref}

\newenvironment{Proof} % имя окружения для доказательства
{\par\noindent{$\blacklozenge$}} % символ рядом с \begin
{\hfill$\scriptstyle\boxtimes$} % символ рядом с \end

\newenvironment{example} % имя окружения для примеров
{\par\noindent{\textsc{\textbf{Пример}.}}} % символ рядом с \begin
{\hfill$\scriptstyle\Box$} % символ рядом с \end

\newenvironment{exercise}
{\par\noindent{\textsc{\textbf{Упражнение}.}}}
{\hfill}

\newtheorem*{theorem}{Теорема} % окружение для теорем
\newtheorem*{corollary}{Следствие} % окружение для следствий
\newtheorem*{lemma}{Лемма} % окружение для лемм
\newtheorem*{mlemma}{M-лемма}
\newtheorem*{theoremp}{Теорема о представлении колебаний}
\newtheorem*{theoremk}{Теорема Кантора}
\newtheorem*{theoremk1}{Теорема Коши}
\newtheorem*{thbv}{Теорема Больцано-Вейерштрасса}
\newtheorem*{cor}{Следствие} % окружение для следствий
\newtheorem*{lem}{Лемма}

\newcommand{\Rm}{\mathbb{R}}
\newcommand{\Cm}{\mathbb{C}}
\newcommand{\I}{\mathbb{I}}
\newcommand{\N}{\mathbb{N}}
\newcommand{\Z}{\mathbb{Z}}
\newcommand{\Q}{\mathbb{Q}}
\newcommand{\F}{\mathcal{F}}
\newcommand{\T}{\mathcal{T}}
\newcommand{\sinx}{\sin x}
% команды для множеств 

\newcommand{\fa}{\forall}
\newcommand{\eps}{\upvarepsilon}
\newcommand{\Ra}{\Rightarrow}
\newcommand{\ra}{\rightarrow}
\newcommand{\lra}{\Longleftrightarrow}
\newcommand{\inRim}{\in\mathcal{R}([a; b])}

\newcommand{\dx}{\text{dx}}
\newcommand{\sgn}{\operatorname{sgn}}

\renewcommand{\ge}{\geqslant}
\renewcommand{\geq}{\geqslant}
\renewcommand{\le}{\leqslant}
\renewcommand{\leq}{\leqslant}

\renewcommand{\omega}{\upomega}
\renewcommand{\tau}{\uptau}
\renewcommand{\beta}{\upbeta}
\renewcommand{\alpha}{\upalpha}
\renewcommand{\nu}{\upnu}
\renewcommand{\mu}{\upmu}
\renewcommand{\psi}{\uppsi}
\renewcommand{\theta}{\Uptheta}
\renewcommand{\phi}{\upvarphi}
\renewcommand{\varphi}{\upvarphi}
\renewcommand{\epsilon}{\upvarepsilon}
\renewcommand{\varepsilon}{\upvarepsilon}
% дополнительные команды
